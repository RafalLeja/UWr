\section{MD5 i jego zastosowania}

\begin{frame}{MD5}
	Funkcja skrótu MD5 (Message-Digest Algorithm 5)
	$h : {0, 1}^* \rightarrow {0, 1}^{128}$
	128-bitowy skrót (hash) z dowolnej długości wejścia.
	128-bitowy stan wewnętrzny
\end{frame}

\begin{frame}{MD5 do przechowywania haseł}
	\begin{itemize}
		\item W latach 1990 - 2000 standard do haszowania haseł
		\item Główna zaleta - szybkość obliczeń
		\item Około 2004 odkryto, że MD5 jest podatne na kolizje.
		\item W rezultacie, wiele systemów zaczęło migrować do bezpieczniejszych algorytmów, takich jak SHA-1* i SHA-256.
		\item*SHA-1 również okazało się podatne na kolizje.
	\end{itemize}
\end{frame}

\begin{frame}{MD5 do przechowywania haseł }
	W 2019 ponad ćwierć CMS-ów używało MD5 do haszowania haseł użytkowników.
\end{frame}

