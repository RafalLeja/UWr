\documentclass[12pt,fleqn]{article}
\usepackage[utf8]{inputenc}
\usepackage[T1]{fontenc}
\usepackage[polish]{babel}
\usepackage{amsmath, amssymb}
\usepackage{geometry}
\usepackage{fancyhdr}
\usepackage{lmodern}
\usepackage{parskip}
\usepackage{systeme}
\usepackage{multicol}

% Page layout
\geometry{margin=1in}
\pagestyle{fancy}
\setlength{\mathindent}{2in}
\fancyhf{}
\lhead{Rafał Leja 340879}
\rhead{RPiS zadanie nr 3}
\cfoot{\thepage}

% Title info
\title{\textbf{Zadanie nr 3}}
\author{Rafał Leja \\
340879 \\
Rachunek prawdopodobieństwa i statystyka}
\date{\today}

\begin{document}

\maketitle

% Example problems
\section*{Dane:}

\begin{align*}
    X_{30}, X_{150}, X_{600} \sim  B(n,p), \text{gdzie } p = 1/3, n = 30, 150, 600 \\
\end{align*}

\section*{Prawdopodobieństwo wprost:}

Prawdopodobieństwo rozkładu Bernoulliego dla $X \sim B(n,p)$ jest dane wzorem:
\begin{align*}
    P(X = k) = \binom{n}{k} p^k (1-p)^{n-k}, \text{ dla } k = 0, 1, \ldots, n
\end{align*}

Więc, dla $X_{n}$ mamy:
\begin{align*}
    P(a \leq  X_n \leq  b) &= \sum_{k=a}^{b} P(X_n = k) \\
    &= \sum_{k=a}^{b} \binom{n}{k} p^k (1-p)^{n-k}
\end{align*}

Co nam daje:
\begin{flalign*}
    &P(8 \leq  X_{30} \leq  12) = \sum_{k=8}^{12} \binom{30}{k} \left(\frac{1}{3}\right)^k \left(\frac{2}{3}\right)^{30-k} \approx 0,66720608 & \\
    &P(40 \leq X_{150} \leq  60) = \sum_{k=40}^{60} \binom{150}{k} \left(\frac{1}{3}\right)^k \left(\frac{2}{3}\right)^{150-k} \approx 0,93151283 \\
    &P(160 \leq  X_{600} \leq  240) = \sum_{k=160}^{240} \binom{600}{k} \left(\frac{1}{3}\right)^k \left(\frac{2}{3}\right)^{600-k} \approx 0,99955211
\end{flalign*}


\section*{Przybliżenia Czebyszewa:}

Wariancja rozkładu Bernoulliego jest dana wzorem:
\begin{align*}
    \sigma^2 = np(1-p)
\end{align*}
a wartość oczekiwana:
\begin{align*}
    \mu = np
\end{align*}

zatem, nierówność Czebyszewa dla rozkładu Bernoulliego jest następująca:

\begin{align*}
    P(|X_n - \mu| \geq k\sigma) &\leq \frac{1}{k^2} \implies P(|X_n - \mu| < k\sigma) \geq 1 - \frac{1}{k^2}
\end{align*}

Dla $X_{30}$ mamy:
\begin{flalign*}
    &\qquad \mu_{30} = 30 \cdot \frac{1}{3} = 10 &\\
    &\qquad \sigma_{30} = \sqrt{30 \cdot \frac{1}{3} \cdot \frac{2}{3}} \approx 2,58 \\
    &\qquad 8 \leq  X_{30} \leq  12 \implies |X_{30} - 10| < 2 \implies k = \frac{2}{2,58} \approx 0,775 \\
    &\qquad P(|X_{30} - 10| < 2) \geq 1 - \frac{1}{(0,775)^2} \approx 1 - 1,65 = -0,65 \text{ (co jest niemożliwe)}
\end{flalign*}

Dla $X_{150}$ mamy analogicznie:
\begin{flalign*}
    &\qquad \space \mu_{150} = 150 \cdot \frac{1}{3} = 50 &\\
    &\qquad \sigma_{150} = \sqrt{150 \cdot \frac{1}{3} \cdot \frac{2}{3}} \approx 6,12 \\
    &\qquad 40 \leq X_{150} \leq 60 \implies |X_{150} - 50| < 10 \implies k = \frac{10}{6,12} \approx 1,63 \\
    &\qquad P(|X_{150} - 50| < 10) \geq 1 - \frac{1}{(1,63)^2} \approx 1 - 0,375 = 0,625
\end{flalign*}

Dla $X_{600}$ mamy:
\begin{flalign*}
    &\qquad \mu_{600} = 600 \cdot \frac{1}{3} = 200 &\\
    &\qquad \sigma_{600} = \sqrt{600 \cdot \frac{1}{3} \cdot \frac{2}{3}} \approx 12,65 \\
    &\qquad 160 \leq  X_{600} \leq  240 \implies |X_{600} - 200| < 40 \implies k = \frac{40}{12,65} \approx 3,16 \\
    &\qquad P(|X_{600} - 200| < 40) \geq 1 - \frac{1}{(3,16)^2} \approx 1 - 0,099 = 0,901
\end{flalign*}

\section*{Przybliżenie normalne:}
Zgodnie z twierdzeniem De Moivre'a-Laplace'a, dla dużych $n$ rozkład $B(n,p)$ można przybliżyć rozkładem normalnym $N(\mu, \sigma^2)$, gdzie:
\begin{align*}
    \mu &= np \\
    \sigma^2 &= np(1-p)
\end{align*}
Zamieniając na rozkład standardowy, mamy:
\begin{align*}
    Z &= \frac{X_n - \mu}{\sigma} \sim N(0,1), \text{wtedy} \\
    P(X_n \leq a) &= P(Z \leq \frac{a - \mu}{\sigma}) = \Phi(\frac{a - \mu}{\sigma}) \\
\end{align*}
jeśli $a$ jest liczbą całkowitą, to musimy dodać 0,5 do $a$ (przybliżenie ciągłe):
\begin{align*}
    P(X_n \leq a) &= P(Z \leq \frac{a + 0,5 - \mu}{\sigma}) \\
\end{align*}

Zatem, dla $X_{30}$ mamy:
\begin{flalign*}
    &\qquad \mu_{30} = 30 \cdot \frac{1}{3} = 10 &\\
    &\qquad \sigma_{30} = \sqrt{30 \cdot \frac{1}{3} \cdot \frac{2}{3}} \approx 2,58 \\
    &\qquad Z = \frac{X_{30} - 10}{2,58} \sim N(0,1) \\
\end{flalign*}
\vspace{-0.5in}
\begin{flalign*}
    \qquad P(8 \leq X_{30} \leq 12) &= P(X_{30} \leq 12) - P(X_{30} \leq 8) &\\
    & = P(Z \leq \frac{12 + 0,5 - 10}{2,58}) - P(Z \leq \frac{8 - 0,5 - 10}{2,58}) \\
    & = \Phi(\frac{2,5}{2,58}) - \Phi(\frac{-2,5}{2,58}) \\
    & = \Phi(0,97) - \Phi(-0,97) \\
    & = 0,834 - 0,165 \\
    & = 0,669
\end{flalign*}

Dla $X_{150}$ mamy:
\begin{flalign*}
    &\qquad \space \mu_{150} = 150 \cdot \frac{1}{3} = 50 &\\
    &\qquad \sigma_{150} = \sqrt{150 \cdot \frac{1}{3} \cdot \frac{2}{3}} \approx 6,12 \\
    &\qquad Z = \frac{X_{150} - 50}{6,12} \sim N(0,1) \\
\end{flalign*}
\vspace{-0.5in}
\begin{flalign*}
    \qquad P(40 \leq X_{150} \leq 60) &= P(X_{150} \leq 60) - P(X_{150} \leq 40) &\\
    & = P(Z \leq \frac{60 + 0,5 - 50}{6,12}) - P(Z \leq \frac{40 - 0,5 - 50}{6,12}) \\
    & = \Phi(\frac{10,5}{6,12}) - \Phi(\frac{-10,5}{6,12}) \\
    & = \Phi(1,71) - \Phi(-1,71) \\
    & = 0,912
\end{flalign*}

Dla $X_{600}$ mamy:
\begin{flalign*}
    &\qquad \mu_{600} = 600 \cdot \frac{1}{3} = 200 &\\
    &\qquad \sigma_{600} = \sqrt{600 \cdot \frac{1}{3} \cdot \frac{2}{3}} \approx 12,65 \\
    &\qquad Z = \frac{X_{600} - 200}{12,65} \sim N(0,1) \\
\end{flalign*}
\vspace{-0.5in}
\begin{flalign*}
    \qquad P(160 \leq X_{600} \leq 240) &= P(X_{600} \leq 240) - P(X_{600} \leq 160) &\\
    & = P(Z \leq \frac{240 + 0,5 - 200}{12,65}) - P(Z \leq \frac{160 - 0,5 - 200}{12,65}) \\
    & = \Phi(\frac{40,5}{12,65}) - \Phi(\frac{-40,5}{12,65}) \\
    & = \Phi(3,20) - \Phi(-3,20) \\
    & = 0,999 \\
\end{flalign*}

\section*{Nierówności Chernoffa:}
Nierówności Chernoffa dla rozkładu Bernoulliego są następujące:
\begin{align*}
    P(X_n \geq a) &\leq e^{-ta}M(t) = e^{-ta} (1-p + pe^t)n \\
\end{align*}

chcemy znaleźć $t$ takie, że $e^{-ta}M(t)$ jest minimalne. Zatem:
\begin{align*}
    \frac{d}{dt} e^{-ta}M(t) &= e^{-ta} (M'(t) - aM(t)) \\
    &= e^{-ta} (1 - p + pe^t)^{n-1} (npe^t - a(1-p+pe^t)) \\
\end{align*}
przyrównując do zera, mamy:
\begin{align*}
    npe^t = a(1-p+pe^t) \\
    npe^t - ape^t = a - ap \\
    e^t = \frac{a - ap}{p(n - a)} \\
\end{align*}
wiemy że $p \neq 0$ oraz $n \neq a$ więc:
\begin{align*}
    t &= \ln\left(\frac{a(1-p)}{p(n - a)}\right) \\
\end{align*}
żeby to rozwiązanie miało sens, musimy mieć $a(1-p) > 0$ oraz $p(n - a) > 0$, co daje nam:
\begin{align*}
    a > 0 \text{ oraz } n - a > 0 \implies 0 < a < n
\end{align*}

Zatem, dla $X_{30}$ mamy:
\begin{flalign*}
    &\qquad \space a = 8, b = 12, n = 30, p = \frac{1}{3} &\\
    &\qquad t_a = \ln\left(\frac{8(1-\frac{1}{3})}{\frac{1}{3}(30 - 8)}\right) \approx -0,3184 \\
    &\qquad t_b = \ln\left(\frac{12(1-\frac{1}{3})}{\frac{1}{3}(30 - 12)}\right) \approx 0,2876\\
\end{flalign*}
\vspace{-0.5in}
\begin{flalign*}
    \qquad P(8 \leq X_{30} \leq 12) &\leq e^{-t_a a}M(t_a) - e^{-t_b b}M(t_b) &\\
    & = e^{-(-0,3184) \cdot 8} \left(1 - \frac{1}{3} + \frac{1}{3} e^{-0,3184}\right)^{30} - e^{-(0,2876) \cdot 12} \left(1 - \frac{1}{3} + \frac{1}{3} e^{0,2876}\right)^{30} \\
    & \approx -0.01502055 \\
    & < 0 \text{ (co jest niemożliwe)}
\end{flalign*}

Dla $X_{150}$ mamy:
\begin{flalign*}
    &\qquad \space a = 40, b = 60, n = 150, p = \frac{1}{3} &\\
    &\qquad t_a = \ln\left(\frac{40(1-\frac{1}{3})}{\frac{1}{3}(150 - 40)}\right) \approx -0,3184 \\
    &\qquad t_b = \ln\left(\frac{60(1-\frac{1}{3})}{\frac{1}{3}(150 - 60)}\right) \approx 0,2876\\
\end{flalign*}
\vspace{-0.5in}
\begin{flalign*}
    \qquad P(40 \leq X_{150} \leq 60) &\leq e^{-t_a a}M(t_a) - e^{-t_b b}M(t_b) &\\
    & = e^{-(-0,3184) \cdot 40} \left(1 - \frac{1}{3} + \frac{1}{3} e^{-0,3184}\right)^{150} - e^{-(0,2876) \cdot 60} \left(1 - \frac{1}{3} + \frac{1}{3} e^{0,2876}\right)^{150} \\
    & \approx -0.02249244 \\
    & < 0 \text{ (co jest niemożliwe)}
\end{flalign*}

Dla $X_{600}$ mamy:
\begin{flalign*}
    &\qquad \space a = 160, b = 240, n = 600, p = \frac{1}{3} &\\
    &\qquad t_a = \ln\left(\frac{160(1-\frac{1}{3})}{\frac{1}{3}(600 - 160)}\right) \approx -0,3184 \\
    &\qquad t_b = \ln\left(\frac{240(1-\frac{1}{3})}{\frac{1}{3}(600 - 240)}\right) \approx 0,2876\\
\end{flalign*}
\vspace{-0.5in}
\begin{flalign*}
    \qquad P(160 \leq X_{600} \leq 240) &\leq e^{-t_a a}M(t_a) - e^{-t_b b}M(t_b) &\\
    & = e^{-(-0,3184) \cdot 160} \left(1 - \frac{1}{3} + \frac{1}{3} e^{-0,3184}\right)^{600} - e^{-(0,2876) \cdot 240} \left(1 - \frac{1}{3} + \frac{1}{3} e^{0,2876}\right)^{600} \\
    & \approx -0.00098321 \\
    & < 0 \text{ (co jest niemożliwe)}
\end{flalign*}

    \end{document}
