\documentclass[12pt,fleqn]{article}
\usepackage[utf8]{inputenc}
\usepackage[T1]{fontenc}
\usepackage[polish]{babel}
\usepackage{amsmath, amssymb}
\usepackage{geometry}
\usepackage{fancyhdr}
\usepackage{lmodern}
\usepackage{parskip}
\usepackage{pgfplots}
\usepackage{systeme}
\usepackage{multicol}

% Page layout
\geometry{margin=1in}
\pagestyle{fancy}
\setlength{\mathindent}{2in}
\fancyhf{}
\lhead{Rafał Leja 340879}
\rhead{RPiS zadanie nr 3}
\cfoot{\thepage}

\makeatletter
\newenvironment{shiftedflalign}{%
  \start@align\tw@%
  \st@rredfalse%
  \m@ne%
  \hskip\parindent%
}{%
  \endalign%
}
\newenvironment{shiftedflalign*}{%
  \start@align\tw@%
  \st@rredtrue%
  \m@ne%
  \hskip\parindent%
}{%
  \endalign%
}
\makeatother

% Title info
\title{\textbf{Zadanie nr 3}}
\author{Rafał Leja \\
340879 \\
Rachunek prawdopodobieństwa i statystyka}
\date{\today}

% plot options
\pgfplotsset{compat=1.18}
\usepgfplotslibrary{external}
\tikzexternalize

\begin{document}

\maketitle

% Example problems
\section*{Dane:}

\begin{align*}
    X_{30}, X_{150}, X_{600} \sim  B(n,p), \text{gdzie } p = 1/3, n = 30, 150, 600 \\
\end{align*}

\section*{Prawdopodobieństwo wprost:}

Prawdopodobieństwo rozkładu Bernoulliego dla $X \sim B(n,p)$ jest dane wzorem:
\begin{align*}
    P(X = k) = \binom{n}{k} p^k (1-p)^{n-k}, \text{ dla } k = 0, 1, \ldots, n
\end{align*}

Więc, dla $X_{n}$ mamy:
\begin{align*}
    P(a \leq  X_n \leq  b) &= \sum_{k=a}^{b} P(X_n = k) \\
    &= \sum_{k=a}^{b} \binom{n}{k} p^k (1-p)^{n-k}
\end{align*}

Co nam daje:
\begin{flalign*}
    &P(8 \leq  X_{30} \leq  12) = \sum_{k=8}^{12} \binom{30}{k} \left(\frac{1}{3}\right)^k \left(\frac{2}{3}\right)^{30-k} \approx 0,66720608 & \\
    &P(40 \leq X_{150} \leq  60) = \sum_{k=40}^{60} \binom{150}{k} \left(\frac{1}{3}\right)^k \left(\frac{2}{3}\right)^{150-k} \approx 0,93151283 \\
    &P(160 \leq  X_{600} \leq  240) = \sum_{k=160}^{240} \binom{600}{k} \left(\frac{1}{3}\right)^k \left(\frac{2}{3}\right)^{600-k} \approx 0,99955211
\end{flalign*}


\section*{Przybliżenia Czebyszewa:}

Wariancja rozkładu Bernoulliego jest dana wzorem:
\begin{align*}
    \sigma^2 = np(1-p)
\end{align*}
a wartość oczekiwana:
\begin{align*}
    \mu = np
\end{align*}

zatem, nierówność Czebyszewa dla rozkładu Bernoulliego jest następująca:

\begin{align*}
    P(|X_n - \mu| \geq k\sigma) &\leq \frac{1}{k^2} \implies P(|X_n - \mu| < k\sigma) \geq 1 - \frac{1}{k^2}
\end{align*}

Dla $X_{30}$ mamy:
\begin{flalign*}
    &\qquad \mu_{30} = 30 \cdot \frac{1}{3} = 10 &\\
    &\qquad \sigma_{30} = \sqrt{30 \cdot \frac{1}{3} \cdot \frac{2}{3}} \approx 2,58 \\
    &\qquad 8 \leq  X_{30} \leq  12 \implies |X_{30} - 10| < 2 \implies k = \frac{2}{2,58} \approx 0,775 \\
    &\qquad P(|X_{30} - 10| < 2) \geq 1 - \frac{1}{(0,775)^2} \approx 1 - 1,65 = -0,65 \text{ (co jest niemożliwe)}
\end{flalign*}

Dla $X_{150}$ mamy analogicznie:
\begin{flalign*}
    &\qquad \space \mu_{150} = 150 \cdot \frac{1}{3} = 50 &\\
    &\qquad \sigma_{150} = \sqrt{150 \cdot \frac{1}{3} \cdot \frac{2}{3}} \approx 6,12 \\
    &\qquad 40 \leq X_{150} \leq 60 \implies |X_{150} - 50| < 10 \implies k = \frac{10}{6,12} \approx 1,63 \\
    &\qquad P(|X_{150} - 50| < 10) \geq 1 - \frac{1}{(1,63)^2} \approx 1 - 0,375 = 0,625
\end{flalign*}

Dla $X_{600}$ mamy:
\begin{flalign*}
    &\qquad \mu_{600} = 600 \cdot \frac{1}{3} = 200 &\\
    &\qquad \sigma_{600} = \sqrt{600 \cdot \frac{1}{3} \cdot \frac{2}{3}} \approx 12,65 \\
    &\qquad 160 \leq  X_{600} \leq  240 \implies |X_{600} - 200| < 40 \implies k = \frac{40}{12,65} \approx 3,16 \\
    &\qquad P(|X_{600} - 200| < 40) \geq 1 - \frac{1}{(3,16)^2} \approx 1 - 0,099 = 0,901
\end{flalign*}

\section*{Przybliżenie normalne:}
Zgodnie z twierdzeniem De Moivre'a-Laplace'a, dla dużych $n$ rozkład $B(n,p)$ można przybliżyć rozkładem normalnym $N(\mu, \sigma^2)$, gdzie:
\begin{align*}
    \mu &= np \\
    \sigma^2 &= np(1-p)
\end{align*}
Zamieniając na rozkład standardowy, mamy:
\begin{align*}
    Z &= \frac{X_n - \mu}{\sigma} \sim N(0,1), \text{wtedy} \\
    P(X_n \leq a) &= P(Z \leq \frac{a - \mu}{\sigma}) = \Phi(\frac{a - \mu}{\sigma}) \\
\end{align*}
jeśli $a$ jest liczbą całkowitą, to musimy dodać 0,5 do $a$ (przybliżenie ciągłe):
\begin{align*}
    P(X_n \leq a) &= P(Z \leq \frac{a + 0,5 - \mu}{\sigma}) \\
\end{align*}

Zatem, dla $X_{30}$ mamy:
\begin{flalign*}
    &\qquad \mu_{30} = 30 \cdot \frac{1}{3} = 10 &\\
    &\qquad \sigma_{30} = \sqrt{30 \cdot \frac{1}{3} \cdot \frac{2}{3}} \approx 2,58 \\
    &\qquad Z = \frac{X_{30} - 10}{2,58} \sim N(0,1) \\
\end{flalign*}
\begin{align*}
    P(8 \leq X_{30} \leq 12) &= P(X_{30} \leq 12) - P(X_{30} \leq 8) \\
    &= P(Z \leq \frac{12 + 0,5 - 10}{2,58}) - P(Z \leq \frac{8 + 0,5 - 10}{2,58}) \\
    &= \Phi(\frac{2,5}{2,58}) - \Phi(\frac{-1,5}{2,58}) \\
    &= \Phi(0,968) - \Phi(-0,581) \\
    &= 0,833 - 0,281 \\
    &= 0,552
\end{align*}
    \end{document}
