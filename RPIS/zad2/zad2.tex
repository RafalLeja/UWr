\documentclass[12pt]{article}
\usepackage[utf8]{inputenc}
\usepackage[T1]{fontenc}
\usepackage[polish]{babel}
\usepackage{amsmath, amssymb}
\usepackage{geometry}
\usepackage{fancyhdr}
\usepackage{lmodern}
\usepackage{parskip}
\usepackage{pgfplots}
\usepackage{systeme}
\usepackage{multicol}

% Page layout
\geometry{margin=1in}
\pagestyle{fancy}
\fancyhf{}
\lhead{Rafał Leja 340879}
\rhead{RPiS zadanie nr 2}
\cfoot{\thepage}

% Title info
\title{\textbf{Zadanie nr 2}}
\author{Rafał Leja \\
340879 \\
Rachunek prawdopodobieństwa i statystyka}
\date{\today}

% plot options
\pgfplotsset{compat=1.18}
\usepgfplotslibrary{external}
\tikzexternalize

\begin{document}

\maketitle

% Example problems
\section*{Dane:}

\begin{align*}
    n &= 8 \\
    m &= 10 \\
\end{align*}

\section*{Trójkąt:}

\begin{multicols}{2}

Rozpatrujemy następujący trójkąt: \\

\begin{tikzpicture}
    \begin{axis}[
        axis equal image,
        xmin=0, xmax=10,
        ymin=0, ymax=12,
        xtick={0,2,...,10},
        ytick={0,2,...,12},
        grid=both,
        xlabel={$x$},
        ylabel={$y$},
        enlargelimits=false
      ]
        \addplot[
          color=blue,
          fill=blue!20,
          thick
        ] coordinates {
          (0,0)
          (8,0)
          (8,10)
          (0,0)
        };
      \end{axis}  \begin{axis}[
        axis equal image,
        xmin=0, xmax=10,
        ymin=0, ymax=12,
        xtick={0,2,...,10},
        ytick={0,2,...,12},
        grid=both,
        xlabel={$x$},
        ylabel={$y$},
        enlargelimits=false
      ]
        \addplot[
          color=blue,
          fill=blue!20,
          thick
        ] coordinates {
          (0,0)
          (8,0)
          (8,10)
          (0,0)
        };
      \end{axis}
\end{tikzpicture}

Trójkąt jest ograniczony prostymi: \\
\begin{align*}
    &\text{1. } x = 0, \\
    &\text{2, } y = 0, \\
    &\text{3, } y = ax + b \\
    &\systeme*{
        0 = a*0 + b,
        10 = a*8 + b
    } \Rightarrow \systeme*{
        b = 0,
        a = \frac{5}{4}
    }
\end{align*}
Zakresy zmiennych to: \\
\begin{align*}
    0 &\leq X \leq 8 \\
    0 &\leq Y \leq \frac{5}{4}x
\end{align*}
\end{multicols}

\vspace{1.5cm}

\section*{Funkcja gęstości (X,Y):}

$f(x,y) = C$ na obszarze trójkąta, gdzie $C$ jest stałą. \\

\begin{align*}
    \int_\mathbb{R}  \int_\mathbb{R}  f(x, y) \, dy \, dx &= 1 \\
    \int_0^8 \int_0^{\frac{5}{4}x} C \, dy \, dx &= 1 \\
    \int_0^8 C \cdot \frac{5}{4}x \, dx &= 1 \\
    C \cdot \frac{5}{8}x^2 \bigg|_0^8 &= 1 \\
    C \cdot \frac{5}{8} \cdot 64 &= 1 \\
    C &= \frac{1}{40} = f(x,y)
\end{align*}

\section*{Zamiana zmiennych:}
Zamieniamy zmienne: \\
\begin{equation*}
    \begin{cases}
        T = X + 2Y \\
        U = X
    \end{cases}
    \Rightarrow 
    \begin{cases}
        X = U \\
        Y = \frac{T - U}{2}
    \end{cases}
\end{equation*}

Obliczamy jakobian: \\
\begin{align*}
    |J| &= \begin{vmatrix}
        \frac{\partial X}{\partial T} & \frac{\partial X}{\partial U} \\
        \frac{\partial Y}{\partial T} & \frac{\partial Y}{\partial U}
    \end{vmatrix} = 
    \begin{vmatrix}
        0 & 1 \\
        \frac{1}{2} & -\frac{1}{2}
    \end{vmatrix} = \frac{1}{2}
\end{align*}

\section*{Nowy zakres zmiennych:}

Zakresy oryginalnych zmiennych: \\
\begin{align*}
    0 &\leq X \leq 8 \\
    0 &\leq Y \leq \frac{5}{4}x
\end{align*}

Zakresy nowych zmiennych: \\
\begin{align*}
    0 &\leq U \leq 8 \\
    0 &\leq \frac{T - U}{2} \leq \frac{5}{4}U
\end{align*}
Inaczej:
\begin{align*}
    0 &\leq \frac{T - U}{2} \leq \frac{5}{4}U \\
    0 &\leq T - U \leq \frac{5}{2}U \\
    U &\leq T \leq U + \frac{5}{2}U \\
    U &\leq T \leq \frac{7}{2}U \\
    0 &\leq T \leq 28
\end{align*}

Zatem: 
\begin{align*}
    &U \in [0, 8]  \cap  [\frac{2}{7}T, T]\\
\end{align*}

\section*{Przedziały całkowania:}
Z powyższych ograniczeń dzielimy obszar na 2 podobszary: \\
\begin{list}{}{}

  \item 1. Dla $T \in [0, 8]$: \\
  \begin{align*}
    &U \in [\frac{2}{7}t, T]\\
  \end{align*}
  więc:
  \begin{align*}
    f_T(t) &= \int f_{X,Y}(x, y) \cdot |J| \, dy \, dx \\
    f_T(t) &= \int_{\frac{2}{7}t}^t \frac{1}{40} \cdot \frac{1}{2} \, du \\
    &= \int_{\frac{2}{7}t}^t \frac{1}{80} \, du \\
    &= \frac{1}{80} \cdot (t - \frac{2}{7}t) \\
    &= \frac{1}{80} \cdot \frac{5}{7}t \\
    &= \frac{1}{112}t
  \end{align*}
\pagebreak
  \item 2. Dla $T \in [8, 28]$: \\
  \begin{align*}
    &U \in [\frac{2}{7}T, 8]\\
  \end{align*}
  więc:
  \begin{align*}
    f_T(t) &= \int f_{X,Y}(x, y) \cdot |J| \, dy \, dx \\
    f_T(t) &= \int_{\frac{2}{7}t}^8 \frac{1}{40} \cdot \frac{1}{2} \, du \\
    &= \int_{\frac{2}{7}t}^8 \frac{1}{80} \, du \\
    &= \frac{1}{80} \cdot (8 - \frac{2}{7}t) \\
    &= \frac{1}{10} - \frac{1}{280}t
  \end{align*}
\end{list}

\section*{Ostateczna funkcja gęstości:}
\begin{align*}
    f_T(t) &= \begin{cases}
        \frac{1}{112}t & \text{dla } 0 \leq t < 8 \\
        \frac{1}{10} - \frac{1}{280}t & \text{dla } 8 \leq t < 28 \\
        0 & \text{dla } t < 0 \text{ lub } t > 28
    \end{cases}
\end{align*}
  \end{document}
