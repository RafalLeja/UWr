\documentclass[12pt]{article}
\usepackage[utf8]{inputenc}
\usepackage[T1]{fontenc}
\usepackage[polish]{babel}
\usepackage{amsmath, amssymb}
\usepackage{geometry}
\usepackage{fancyhdr}
\usepackage{lmodern}
\usepackage{parskip}
\usepackage{pgfplots}

% Page layout
\geometry{margin=1in}
\pagestyle{fancy}
\fancyhf{}
\lhead{Rafał Leja 340879}
\rhead{RPiS zadanie nr 2}
\cfoot{\thepage}

% Title info
\title{\textbf{Zadanie nr 2}}
\author{Rafał Leja \\
340879 \\
Rachunek prawdopodobieństwa i statystyka}
\date{\today}

% plot options
\pgfplotsset{compat=1.18}
\usepgfplotslibrary{external}
\tikzexternalize

\begin{document}

\maketitle

% Example problems
\section*{Dane:}

\begin{align*}
    n &= 8 \\
    m &= 10 \\
\end{align*}

\section*{Gęstość X, Y:}

Rozpatrujemy następujący trójkąt: \\

\begin{tikzpicture}
    \begin{axis}[
        axis equal image,
        xmin=0, xmax=10,
        ymin=0, ymax=12,
        xtick={0,2,...,10},
        ytick={0,2,...,12},
        grid=both,
        xlabel={$x$},
        ylabel={$y$},
        enlargelimits=false
      ]
        \addplot[
          color=blue,
          fill=blue!20,
          thick
        ] coordinates {
          (0,0)
          (8,0)
          (8,10)
          (0,0)
        };
      \end{axis}  \begin{axis}[
        axis equal image,
        xmin=0, xmax=10,
        ymin=0, ymax=12,
        xtick={0,2,...,10},
        ytick={0,2,...,12},
        grid=both,
        xlabel={$x$},
        ylabel={$y$},
        enlargelimits=false
      ]
        \addplot[
          color=blue,
          fill=blue!20,
          thick
        ] coordinates {
          (0,0)
          (8,0)
          (8,10)
          (0,0)
        };
      \end{axis}
\end{tikzpicture}

trójkąt jest ograniczony prostymi: \\
\begin{align*}
    x &= 0 \\
    y &= 0 \\
    y &= \frac{10}{8}x \\
\end{align*}

\vspace{1cm}

% Add more problems as needed
% \section*{Problem 3}
% ...

\end{document}
