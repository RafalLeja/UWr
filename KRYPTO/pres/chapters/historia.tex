\section{Standardy sieci komórkowych}

\begin{frame}{1G - Początki}
	\alert{\textbf{NTT}}, Nippon Telegraph and Telephone - \textbf{1979}, \textbf{Japonia}

	\alert{\textbf{NMT}}, Nordic Mobile Telephone - \textbf{1981}, \textbf{Skandynawia}

	\alert{\textbf{AMPS}}, Advanced Mobile Phone System - \textbf{1983}, \textbf{USA}

\end{frame}

\begin{frame}{1G}
	\begin{itemize}
		\item \textbf{Analogowe} systemy telefonii komórkowej
		\item \textbf{Tylko głos}, brak transmisji danych
		\item \textbf{Brak szyfrowania}, podatność na podsłuch
	\end{itemize}
\end{frame}

\begin{frame}{2G - Cyfrowa rewolucja}
	\alert{\textbf{GSM}}, Global System for Mobile Communications - \textbf{1991}, \textbf{Europa}

	\alert{\textbf{CDMA}}, Code Division Multiple Access - \textbf{1995}, \textbf{USA}

	\alert{\textbf{PDC}}, Personal Digital Cellular - \textbf{1993}, \textbf{Japonia}
\end{frame}

\begin{frame}{2G}
	\begin{itemize}
		\item \textbf{Cyfrowe} systemy telefonii komórkowej
		\item Wprowadzenie \textbf{transmisji danych} (SMS, MMS)
		\item \textbf{Szyfrowanie}:
		      \begin{itemize}
			      \item \alert{\textbf{A5/1}} - silne szyfrowanie, używane w Europie
			      \item \alert{\textbf{A5/2}} - słabsze, eksportowe szyfrowanie
			      \item \alert{\textbf{A5/3}} (KASUMI) - ulepszona wersja, stosowana w późniejszych implementacjach GSM
		      \end{itemize}
	\end{itemize}
\end{frame}

\begin{frame}{3G - Era multimediów}
	\alert{\textbf{UMTS}}, Universal Mobile Telecommunications System - \textbf{2001}, \textbf{Europa}

	\alert{\textbf{CDMA2000}} - \textbf{2000}, \textbf{USA}

	\alert{\textbf{HSPA/HSPA+}} - \textbf{2005/2007}, \textbf{Globalnie}
\end{frame}

\begin{frame}{3G}
	\begin{itemize}
		\item \textbf{Szybsze prędkości transmisji danych} (384 Kbps / 42 Mbps)
		\item Obsługa \textbf{multimediów} (wideo, VoIP)
		\item Ulepszone \textbf{szyfrowanie}:
		      \begin{itemize}
			      \item \alert{\textbf{KASUMI}} - używany w UMTS, oparty na A5/3
			      \item \alert{\textbf{SNOW 3G}} - używany w CDMA2000, oparty na SNOW 2.0
		      \end{itemize}
		\item \textbf{Integralność} danych i \textbf{uwierzytelnianie} użytkowników
	\end{itemize}
\end{frame}

\begin{frame}{4G - Era LTE}
	\alert{\textbf{LTE}}, Long Term Evolution - \textbf{2009}, \textbf{Globalnie}

	\alert{\textbf{WiMAX}} - \textbf{2007}, \textbf{Globalnie}
\end{frame}

\begin{frame}{4G}
	\begin{itemize}
		\item Standard \textbf{IP} dla transmisji danych i głosu (\textbf{VoLTE})
		\item \textbf{Bardzo szybkie} prędkości transmisji danych (100 Mbps / 1 Gbps)
		\item Obsługa \textbf{zaawansowanych multimediów} (HD wideo, gry online)
		\item Zaawansowane \textbf{szyfrowanie}:
		      \begin{itemize}
			      \item \alert{\textbf{AES}} - używany w LTE, oparty na standardzie AES
			      \item \alert{\textbf{SNOW 3G}} - kontynuacja użycia z 3G
		      \end{itemize}
		\item Ulepszone \textbf{uwierzytelnianie} i \textbf{integralność} danych
	\end{itemize}
\end{frame}

\begin{frame}{5G - Era łączności masowej}
	\alert{\textbf{5G NR}}, New Radio - \textbf{2019}, \textbf{Globalnie}

	\alert{\textbf{5G mmWave}} - \textbf{2019}, \textbf{Globalnie}
\end{frame}

\begin{frame}{5G}
	\begin{itemize}
		\item Jeszcze szybsze prędkości (\textbf{10 Gbps} / \textbf{20 Gbps})
		\item Bardzo niskie opóźnienia (\textbf{1 ms})
		\item Obsługa \textbf{masowej łączności} (IoT, mMTC)
		\item Szyfrowanie i bezpieczeństwo:
		      \begin{itemize}
			      \item \alert{\textbf{AES-256}} - wzmocnione szyfrowanie dla 5G
			      \item \alert{\textbf{ZUC}} - algorytm strumieniowy używany w 5G, oparty na ZUC-128
			      \item \textbf{SUCI} zamiast IMSI
		      \end{itemize}
	\end{itemize}
\end{frame}

