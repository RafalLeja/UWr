\section{Standardy sieci komórkowych}

\begin{frame}{1G - Początki}
	\alert{\textbf{NTT}}, Nippon Telegraph and Telephone - \textbf{1979}, \textbf{Japonia}

	\alert{\textbf{NMT}}, Nordic Mobile Telephone - \textbf{1981}, \textbf{Skandynawia}

	\alert{\textbf{AMPS}}, Advanced Mobile Phone System - \textbf{1983}, \textbf{USA}

\end{frame}

\begin{frame}{1G}
	\begin{itemize}
		\item \textbf{Analogowe} systemy telefonii komórkowej
		\item Tylko głos, brak transmisji danych
		\item Brak szyfrowania, podatność na podsłuch
	\end{itemize}
\end{frame}

\begin{frame}{2G - Cyfrowa rewolucja}
	\alert{\textbf{GSM}}, Global System for Mobile Communications - \textbf{1991}, \textbf{Europa}

	\alert{\textbf{CDMA}}, Code Division Multiple Access - \textbf{1995}, \textbf{USA}

	\alert{\textbf{PDC}}, Personal Digital Cellular - \textbf{1993}, \textbf{Japonia}
\end{frame}

\begin{frame}{2G}
	\begin{itemize}
		\item \textbf{Cyfrowe} systemy telefonii komórkowej
		\item Wprowadzenie transmisji danych (SMS, MMS)
		\item Podstawowe szyfrowanie, ale nadal podatne na ataki
	\end{itemize}
\end{frame}
% \begin{frame}{Historia sieci komórkowych}
% 	\begin{itemize}
% 		\item \textbf{1G} - lata 80. i wczesne 90.)
%
% 		\item \textbf{2G} - druga generacja sieci komórkowych (lata 90.)
%
% 		\item \textbf{3G} - trzecia generacja sieci komórkowych (początek lat 2000.)
%
% 		\item \textbf{4G} - czwarta generacja sieci komórkowych (od około 2010 roku)
%
% 		\item \textbf{5G} - piąta generacja sieci kom
%
% 	\end{itemize}
% \end{frame}
%
