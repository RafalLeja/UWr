\section{Standardy sieci komórkowych}

\begin{frame}{1G - Początki}
	\alert{\textbf{NTT}}, Nippon Telegraph and Telephone - \textbf{1979}, \textbf{Japonia}

	\alert{\textbf{NMT}}, Nordic Mobile Telephone - \textbf{1981}, \textbf{Skandynawia}

	\alert{\textbf{AMPS}}, Advanced Mobile Phone System - \textbf{1983}, \textbf{USA}

\end{frame}

\begin{frame}{1G}
	\begin{itemize}
		\item \textbf{Analogowe} systemy telefonii komórkowej
		\item \textbf{Tylko głos}, brak transmisji danych
		\item \textbf{Brak szyfrowania}, podatność na podsłuch
	\end{itemize}
\end{frame}

\begin{frame}{2G - Cyfrowa rewolucja}
	\alert{\textbf{GSM}}, Global System for Mobile Communications - \textbf{1991}, \textbf{Europa}

	\alert{\textbf{CDMA}}, Code Division Multiple Access - \textbf{1995}, \textbf{USA}

	\alert{\textbf{PDC}}, Personal Digital Cellular - \textbf{1993}, \textbf{Japonia}
\end{frame}

\begin{frame}{2G}
	\begin{itemize}
		\item \textbf{Cyfrowe} systemy telefonii komórkowej
		\item Wprowadzenie \textbf{transmisji danych} (SMS, MMS)
		\item \textbf{Szyfrowanie}:
		      \begin{itemize}
			      \item \alert{\textbf{A5/1}} - silne szyfrowanie, używane w Europie
			      \item \alert{\textbf{A5/2}} - słabsze, eksportowe szyfrowanie
			      \item \alert{\textbf{A5/3}} (KASUMI) - ulepszona wersja, stosowana w późniejszych implementacjach GSM
		      \end{itemize}
	\end{itemize}
\end{frame}

\begin{frame}{3G - Era multimediów}
	\alert{\textbf{UMTS}}, Universal Mobile Telecommunications System - \textbf{2001}, \textbf{Europa}

	\alert{\textbf{CDMA2000}} - \textbf{2000}, \textbf{USA}

	\alert{\textbf{WCDMA}} - Wideband Code Division Multiple Access, używany w UMTS
\end{frame}
