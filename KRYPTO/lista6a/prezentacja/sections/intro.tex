\section{Dawno dawno temu w internecie...}

\begin{frame}{Scenariusz: postacie}
	\begin{tikzpicture}[
		node distance=36mm and 40mm,
		box/.style={draw, rounded corners=3pt, minimum width=38mm, minimum height=12mm, align=center, fill=white},
		arrow/.style={-Latex, thick},
		dashedarrow/.style={->, dashed}
	]

	% Nodes
	\node[box] (client) {Web Browser\\(Client)};
	\node[box, right=of client] (internet) {Internet / Router};
	\node[box, right=of internet] (server) {Web Server\\(port 80)};

	% DNS resolver below internet
	\node[box, below=18mm of internet] (dns) {DNS Resolver};

	% Labels
	\node[below=4mm of client.south, align=center] (cnote) {};
	\node[below=4mm of server.south, align=center] (snote) {};

	% Arrows: DNS lookup
	\draw[arrow] (client.south east) .. controls +(down:8mm) and +(left:8mm) .. (dns.north west) node[midway, below left]{DNS query (UDP 53)};
	\draw[arrow] (dns.north east) .. controls +(right:8mm) and +(down:8mm) .. (client.south west) node[midway, below right]{DNS response (A record)};

	% TCP handshake (three-way)
	\draw[arrow] (client.east) -- ++(8mm,0) node[midway,above]{TCP SYN (to :80)} -- ++(20mm,0) |- (server.west);
	\draw[arrow] (server.west) -- ++(-8mm,0) node[midway,below]{SYN/ACK} -- ++(-20mm,0) |- (client.east);
	\draw[arrow] (client.east) -- ++(24mm,0) node[midway,above]{ACK} -- ++(24mm,0) |- (server.west);

	% Simplified straight arrows for clarity (HTTP request/response)
	\draw[arrow] (client) -- (server) node[midway, above] {HTTP GET / (port 80)(unencrypted)};
	\draw[arrow] (server) -- (client) node[midway, below] {HTTP/1.1 200 OKHTML, images, etc. (unencrypted)};

	% Security warning brace
	\draw[decorate,decoration={brace,mirror, amplitude=6pt}] ($(client.south)+(0,-14mm)$) -- ($(server.south)+(0,-14mm)$) node[midway,below=10pt, align=center] {\textbf{Plain HTTP — traffic is not encrypted}\\(vulnerable to eavesdropping / MITM)};

	% Notes
	\node[align=left, right=6mm of server, xshift=4mm] (notes) {
		\small\textbf{Legend:}\\
		\small DNS = name $\to$ IP,\; TCP 3-way handshake establishes connection,\\
		\small HTTP over port 80 is transmitted in cleartext.
	};

\end{tikzpicture}

\end{frame}

\begin{frame}{Scenariusz: odwiedzanie strony www w 1994 r.}
	Chcemy wejść na strone www.example.com.
	\begin{itemize}
		\item<1-> Przeglądarka: Gdzie jest www.example.com? -> DNS
		\item<1-> Wpisujemy w przeglądarce adres URL i naciskamy Enter.
		\item<2-> Przeglądarka musi znaleźć adres IP serwera, na którym znajduje się strona. Wysyła zapytanie do DNS.
		\item<3-> Otrzymuje odpowiedź z adresem IP i łączy się z serwerem na porcie 80.
		\item<4-> Wysyła żądanie HTTP o stronę www.example.com.
		\item<5-> Serwer odpowiada stroną HTML, którą przeglądarka renderuje i wyświetla użytkownikowi.
	\end{itemize}
\end{frame}

\begin{frame}{Scenariusz: Ewa przekieruje ruch do swojego serwera}
	Przed atakem Ewa musi się przygotować.
	\begin{itemize}
		\item<1-> Ewa rejestruje domenę www.eve.com i ustawia jej rekord A na swój serwer.
		\item<2-> Ewa modyfikuje rekordy DNS dla www.example.com, aby wskazywały na jej serwer (np. poprzez atak DNS spoofing).
	\end{itemize}

\end{frame}
\begin{frame}{Scenariusz: odwiedzanie strony www po ataku Ewy}
	Chcemy wejść na strone www.example.com.
	\begin{itemize}
		\item Wpisujemy w przeglądarce adres URL i naciskamy Enter.
		\item Przeglądarka musi znaleźć adres IP serwera, na którym znajduje się strona. Wysyła zapytanie do DNS.
		\item<3-> Ewa odpowiada, podając adres IP swojego serwera.
		\item<4-> Wysyła żądanie HTTP o stronę www.example.com.
		\item<5-> Serwer Ewy odpowiada stroną HTML, która może wyglądać jak oryginalna strona, ale jest fałszywa.
	\end{itemize}
\end{frame}

\begin{frame}{Słabe punkty scenariusza}
	\begin{itemize}
		\item<1-> Brak uwierzytelniania serwera DNS: przeglądarka ufa odpowiedzi DNS bez weryfikacji.
		      \begin{itemize}
			      \item<2-> Współczesne rozwiązanie: DNSSEC.
			      \item<3-> Problem: DNSSEC nie jest powszechnie wdrożony.
		      \end{itemize}
		\item<4-> Brak uwierzytelniania serwera WWW: przeglądarka nie sprawdza, czy serwer WWW jest tym, za kogo się podaje.
		      \begin{itemize}
			      \item<5-> Współczesne rozwiązanie: SSL/TLS z certyfikatami X.509.
		      \end{itemize}
	\end{itemize}
\end{frame}
