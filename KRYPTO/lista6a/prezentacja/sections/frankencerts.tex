\section{Frankencerts}
\begin{frame}{Frankencerts}
	\begin{itemize}
		\item Frankencerts to podejście do automatycznego testowania implementacji SSL/TLS.
		\item Składa się z dwóch głównych komponentów:
		      \begin{itemize}
			      \item Generatora certyfikatów z kontrolowanymi błędami.
			      \item Wyroczni do weryfikacji certyfikatów.
		      \end{itemize}
	\end{itemize}
\end{frame}

\begin{frame}{Generowanie Frankencerts}
	\begin{itemize}
		\item Zbieramy prawdziwe certyfikaty (243,246)
		\item<2-> Dzielimy je na części składowe (pola, rozszerzenia)
		\item<3-> Miksujemy części z różnych certyfikatów, tworząc nowe certyfikaty z kontrolowanymi błędami
	\end{itemize}

\end{frame}
\begin{frame}{Generowanie Frankencerts}
	\centering
	\includegraphics[height=200pt]{images/frankencert-alg1.png}
\end{frame}

\begin{frame}{Generowanie Frankencerts}
	\centering
	\includegraphics[height=150pt]{images/frankencert-alg2.png}
\end{frame}

\begin{frame}{Generowanie Frankencerts}
	Rezultat?\\
	8,127,600 przetestowanych certyfikatów z różnymi kombinacjami błędów!\\
\end{frame}

\begin{frame}{Wyrocznia}
	\begin{itemize}
		\item Testowanie ``różnicowe''
		\item Testujemy wiele implementacji SSL/TLS na raz
		\item Dla każdego wygenerowanego Frankencert, sprawdzamy czy wszystkie implementacje zgadzają się co do jego ważności
	\end{itemize}
\end{frame}

\begin{frame}{Wyrocznia}
	Testowane implementacje:
	\begin{multicols}{2}
		\begin{itemize}
			\item OpenSSL 1.0.1e
			\item PolarSSL 1.2.8
			\item GnuTLS 3.1.9.1
			\item CyaSSL 2.7.0
			\item MatrixSSL 3.4.2
			\item NSS 3.15.2
			\item cryptlib 3.4.0-r1
			\item OpenJDK 1.7.0 09-b30
			\item Bouncy Castle 1.49
			\item Firefox 20.0
			\item Chrome 30.0.1599.114 p1
			\item WebKitGTK 1.10.2-r300
			\item Opera 12.0
			\item Safari 7.0
			\item IE 10.0
		\end{itemize}
	\end{multicols}
\end{frame}
