%! TEX root = ./pres.tex
\documentclass{beamer}
\usepackage[utf8]{inputenc}
\usepackage[polish]{babel}
\usepackage{tikz}
\usepackage[T1]{fontenc}
\usepackage{graphicx}
\usepackage{multicol}

\usetheme{Warsaw}
\usecolortheme{beaver}
\usetikzlibrary{trees,positioning,arrows.meta,decorations.pathreplacing}
% \usepackage{twi_theme}

\title{Kryptoanaliza stosowana}
\subtitle{Frankencerts - Jak testować certyfikaty SSL/TLS}
\author{Rafał Leja (UWr)}
\date{26 listopada 2025}

% -----------------------------------------
% DOCUMENT
% -----------------------------------------
\begin{document}

% -----------------------------------------
% TITLE SLIDE
% -----------------------------------------
\begin{frame}
	\titlepage
\end{frame}

% \section{intro}

\begin{frame}{Plan}
	\begin{enumerate}
		\item Tło: MD5 i problem do rozwiązania.
		\item Implementacja: CPU vs GPU.
		\item Benchmarki.
		\item Podsumowanie.
	\end{enumerate}
\end{frame}


\section{SSL/TLS}

\begin{frame}{SSL/TLS - Podstawy}
	\begin{itemize}
		\item<1-> SSL (Secure Sockets Layer) i jego następca TLS (Transport Layer Security) to protokoły kryptograficzne zapewniające bezpieczną komunikację w sieci.
		\item<2-> Główne cele SSL/TLS:
		      \begin{itemize}
			      \item<3-> Poufność: szyfrowanie danych przesyłanych między klientem a serwerem.
			      \item<4-> Integralność: zapewnienie, że dane nie zostały zmienione podczas transmisji.
			      \item<5-> Uwierzytelnianie: weryfikacja tożsamości serwera (i opcjonalnie klienta) za pomocą certyfikatów cyfrowych.
		      \end{itemize}
		\item<6-> SSL/TLS jest szeroko stosowany w protokołach takich jak HTTPS, SMTP, FTP itp.
	\end{itemize}
\end{frame}

\section{Testowanie implementacji SSL/TLS}

\begin{frame}{Jak testować poprawność implementacji}
	\begin{itemize}
		\item Skąd brać różne certyfikaty do testów?
		      \pause{}
		      \begin{itemize}
			      \item Prawdziwe certyfikaty z Internetu $\Rightarrow$ Tylko poprwane certyfikaty.
			      \item Fuzzing $\Rightarrow$ Większość certyfikatów będzie odrzucona na poziomie parsowania.
			      \item Manualne tworzenie certyfikatów $\Rightarrow$ Czasochłonne.
		      \end{itemize}
		\item Potrzebne jest automatyczne generowanie certyfikatów z kontrolowanymi błędami.
	\end{itemize}
\end{frame}

\begin{frame}{Jak testować poprawność implementacji}
	\begin{itemize}
		\item Jak interpretować wyniki testów?
		      \pause{}
		      \begin{itemize}
			      \item Manualna analiza wyników $\Rightarrow$ Czasochłonna i podatna na błędy.
			      \item Automatyczna analiza wyników $\Rightarrow$ Wymaga bezbłędnego zaimplementowania SSL/TLS.
		      \end{itemize}
		\item Potrzebujemy  ``Wyroczni'' weryfikującej certyfikaty.
	\end{itemize}
\end{frame}

\section{Frankencerts}
\begin{frame}{Frankencerts}
	\begin{itemize}
		\item Frankencerts to podejście do automatycznego testowania implementacji SSL/TLS.
		\item Składa się z dwóch głównych komponentów:
		      \begin{itemize}
			      \item Generatora certyfikatów z kontrolowanymi błędami.
			      \item Wyroczni do weryfikacji certyfikatów.
		      \end{itemize}
	\end{itemize}
\end{frame}

\begin{frame}{Generowanie Frankencerts}
	\begin{itemize}
		\item Zbieramy prawdziwe certyfikaty (243,246)
		\item<2-> Dzielimy je na części składowe (pola, rozszerzenia)
		\item<3-> Miksujemy części z różnych certyfikatów, tworząc nowe certyfikaty z kontrolowanymi błędami
	\end{itemize}

\end{frame}
\begin{frame}{Generowanie Frankencerts}
	\centering
	\includegraphics[height=200pt]{images/frankencert-alg1.png}
\end{frame}

\begin{frame}{Generowanie Frankencerts}
	\centering
	\includegraphics[height=150pt]{images/frankencert-alg2.png}
\end{frame}

\begin{frame}{Generowanie Frankencerts}
	Rezultat?\\
	8,127,600 przetestowanych certyfikatów z różnymi kombinacjami błędów!\\
\end{frame}

\begin{frame}{Wyrocznia}
	\begin{itemize}
		\item Testowanie ``różnicowe''
		\item Testujemy wiele implementacji SSL/TLS na raz
		\item Dla każdego wygenerowanego Frankencert, sprawdzamy czy wszystkie implementacje zgadzają się co do jego ważności
	\end{itemize}
\end{frame}

\begin{frame}{Wyrocznia}
	Testowane implementacje:
	\begin{multicols}{2}
		\begin{itemize}
			\item OpenSSL 1.0.1e
			\item PolarSSL 1.2.8
			\item GnuTLS 3.1.9.1
			\item CyaSSL 2.7.0
			\item MatrixSSL 3.4.2
			\item NSS 3.15.2
			\item cryptlib 3.4.0-r1
			\item OpenJDK 1.7.0 09-b30
			\item Bouncy Castle 1.49
			\item Firefox 20.0
			\item Chrome 30.0.1599.114 p1
			\item WebKitGTK 1.10.2-r300
			\item Opera 12.0
			\item Safari 7.0
			\item IE 10.0
		\end{itemize}
	\end{multicols}
\end{frame}

\section{Wyniki}
\begin{frame}{Wyniki}
	\begin{itemize}
		\item 8,127,600 przetestowanych certyfikatów
		\item Generowanie oraz testowanie certyfikatów: $\approx$ 11.75 dni\\
		\item 208 unikatowych kombinacji błędów
		\item  62,022 certyfikatów powodujących niezgodności

	\end{itemize}
\end{frame}

\begin{frame}{Wyniki}
	\centering
	\includegraphics[height=200pt]{images/result-table1.png}
\end{frame}

\begin{frame}{Wyniki}
	\centering
	\includegraphics[height=70pt]{images/result-table2.png}
\end{frame}


\begin{frame}{Źródła}
	\begin{itemize}
		\item \href{https://ieeexplore.ieee.org/stamp/stamp.jsp?tp=&arnumber=6956560}{C. Brubaker, S. Jana, B. Ray, S. Khurshid, and V. Shmatikov. Using Frankencerts for automated adversarial testing of certificate validation in SSL/TLS implementations. In 35th IEEE Symposium on Security and Privacy, pages 114–129. IEEE Computer Society, 2014.}
		\item \href{https://en.wikipedia.org/wiki/Transport_Layer_Security}{Wikipedia: Transport Layer Security}
		\item \href{https://en.wikipedia.org/wiki/X.509}{Wikipedia: X.509}
		\item \href{https://www.cloudflare.com/learning/ssl/what-happens-in-a-tls-handshake/}{Cloudflare: What is a TLS handshake?}
		\item \href{https://www.cloudflare.com/learning/ssl/what-is-an-ssl-certificate/}{Cloudflare: What is an SSL certificate?}
	\end{itemize}
\end{frame}

\end{document}
